%%%%%%%%%%%%%%%%%%%%%%%%%%%%%%%%%%%%%%%%%
% Их сургуулийн оюутны тезис  
% LaTeX Загвар
% Version 2.3 (25/3/16)
%
% Энэ загвар нь дараах сайтаас авсан загварын монгол хувилбар юм.
% http://www.LaTeXTemplates.com
%
% Version 2.x major modifications by:
% Vel (vel@latextemplates.com)
%
% Анхдагч загварын эх үүсвэр:
% Steve Gunn (http://users.ecs.soton.ac.uk/srg/softwaretools/document/templates/)
% Sunil Patel (http://www.sunilpatel.co.uk/thesis-template/)
%
% Загварын лиценз:
% CC BY-NC-SA 3.0 (http://creativecommons.org/licenses/by-nc-sa/3.0/)
%
%%%%%%%%%%%%%%%%%%%%%%%%%%%%%%%%%%%%%%%%%

%-------------------------------------------------------------------------------
%	PACKAGES AND OTHER DOCUMENT CONFIGURATIONS
%-------------------------------------------------------------------------------

\documentclass[
12pt, % Баримтын фонтын хэмжээ, сонголт: 10pt, 11pt, 12pt
oneside, % Хоёр талаар хэвлэж үдэхээр тохируулсан. Нэг тал бол комментыг арилга
%chapterinoneline,% Нэг мөрөнд бүлгийн дугаар, нэрийг гаргах
english, % babel багцын хэлний тохиргоо
onehalfspacing, % Мөр хоорондын зай. Сонголтууд: singlespacing, onehalfspacing, doublespacing
%draft, % Ноорог горимд шилжихийн тулд комментыг арилга(зураг, холбоос, hboxes гарахгүй)
nolistspacing, % Хэрэв мөр хоорондын зай onehalfspacing эсвэл doublespacing бол, жагсаалтын мөр хоорондын зайг single болгохын тулд комментыг арилга
%liststotoc, % Зураг/хүснэгт/бусад жагсаалтыг гарчигт оруулахын тулд комментыг арилга
%toctotoc, % Uncomment to add the main table of contents to the table of contents
%parskip, % Параграф хооронд зай оруулахын тулд комментыг арилга
%nohyperref, % hyperref багцыг ачаалахгүй бол комментыг арилга
headsepline, % Толгой мөрийн доогуур шугам татахын тулд комментыг арилга
]{MUST-Thesis} % Энэ класс файл нь баримтын бүтцийг тодорхойлно

\usepackage[utf8]{inputenc} % Олон улсын тэмдэгт оруулахад хэрэгтэй
\usepackage[T2A]{fontenc} % Олон улсын тэмдэгтийн гаралтын кодчилол
\usepackage[mongolian]{babel}

\usepackage{titlesec}
\usepackage[table]{xcolor}
\usepackage{rotating} % эргүүлэх
\usepackage{ragged2e}
\let\oldtabular\tabular
\let\endoldtabular\endtabular
\renewenvironment{tabular}
   {\bgroup\singlespacing\oldtabular}%
   {\endoldtabular\egroup}
\let\oldverse\verse
\let\endoldverse\endverse
\renewenvironment{verse}
   {\bgroup\singlespacing\oldverse}%
   {\endoldverse\egroup}

\usepackage{caption,longtable,ltcaption}
\DeclareCaptionJustification{nohyphen}{\hyphenpenalty=10000}
\captionsetup{justification=nohyphen}

\usepackage{pgf}
\usepackage{tikz} % зураг зурах
\usetikzlibrary{shapes,arrows,automata}

\definecolor{OliveGreen}{cmyk}{0.64,0,0.95,0.40}
\definecolor{CadetBlue}{cmyk}{0.62,0.57,0.23,0}
\definecolor{Cover}{RGB}{255,255,255}
% \definecolor{Cover}{RGB}{102,255,255}
% \definecolor{Chapter}{RGB}{255,220,180}

\definecolor{lightlightgray}{gray}{0.9}
% \usepackage{listings}
\usepackage{listings,lstautogobble}
\renewcommand{\lstlistingname}{Эх код} % Програмын эх код хэвлэх
\lstset{
    language=C,                             % Code langugage
    basicstyle=\linespread{1.0}\ttfamily,                   % Code font
    keywordstyle=\color{OliveGreen},        % Keywords font ('*' = uppercase)
    commentstyle=\color{CadetBlue},         % Comments font
    numbers=left,                           % Line nums position
    numberstyle=\tiny,                      % Line-numbers fonts
    stepnumber=1,                           % Step between two line-numbers
    numbersep=5pt,                          % How far are line-numbers from code
    backgroundcolor=\color{lightlightgray}, % Choose background color
    frame=none,                             % A frame around the code
    tabsize=2,                              % Default tab size
    captionpos=t,                           % Caption-position = bottom
    breaklines=true,                        % Automatic line breaking?
    breakatwhitespace=false,                % Automatic breaks only at whitespace?
    showspaces=false,                       % Dont make spaces visible
    showtabs=false,                         % Dont make tabls visible
    columns=flexible,                       % Column format
    morekeywords={__global__, __device__},  % CUDA specific keywords
}

% source code
\colorlet{punct}{red!60!black}
\definecolor{background}{HTML}{EEEEEE}
\definecolor{delim}{RGB}{20,105,176}
\colorlet{numb}{magenta!60!black}
\lstdefinelanguage{json}{
    basicstyle=\normalfont\ttfamily,
    numbers=left,
    numberstyle=\scriptsize,
    stepnumber=1,
    numbersep=8pt,
    showstringspaces=false,
    breaklines=true,
    frame=lines,
    autogobble=true,
    backgroundcolor=\color{background},
    literate=
     *{0}{{{\color{numb}0}}}{1}
      {1}{{{\color{numb}1}}}{1}
      {2}{{{\color{numb}2}}}{1}
      {3}{{{\color{numb}3}}}{1}
      {4}{{{\color{numb}4}}}{1}
      {5}{{{\color{numb}5}}}{1}
      {6}{{{\color{numb}6}}}{1}
      {7}{{{\color{numb}7}}}{1}
      {8}{{{\color{numb}8}}}{1}
      {9}{{{\color{numb}9}}}{1}
      {:}{{{\color{punct}{:}}}}{1}
      {,}{{{\color{punct}{,}}}}{1}
      {\{}{{{\color{delim}{\{}}}}{1}
      {\}}{{{\color{delim}{\}}}}}{1}
      {[}{{{\color{delim}{[}}}}{1}
      {]}{{{\color{delim}{]}}}}{1},
}

\usepackage[autostyle=false]{csquotes} % Ном зүйд хэлнээс хамаарсан хашилт оруулахад хэрэгтэй

\usepackage[backend=bibtex,natbib=true,sorting=none,sortcites]{biblatex} % Ном зүйд bibtex -г ашиглах

\usepackage{float}
\usepackage{geometry}
\usepackage{graphicx}
\usepackage{subcaption}
\usepackage{hyperref}
\usepackage{indentfirst}
\usepackage{enumitem}
\usepackage{textcase}
\usepackage{tocloft}
\usepackage{multicol}
\usepackage{pdflscape}
\usepackage{etoolbox}
\usepackage{amsmath}
\usepackage{multirow}
\usepackage{setspace}
\graphicspath{ {./} }


\newcommand{\authorshipname}{Зохиогч эрхийн хамгаалал}
\newcommand{\abbrevname}{Товчилсон үгс}
\newcommand{\constantsname}{Ашигласан томьёонууд}
\newcommand{\symbolsname}{Таних тэмдэгтүүд}
\newcommand{\acknowledgementname}{Талархал}
\newcommand{\abstractname}{Хураангуй}
 
%-------------------------------------------------------------------------------
%	THESIS INFORMATION
%-------------------------------------------------------------------------------

\thesistitle{Шаблом засах} % Таны ажлын нэр, нүүр болон хураангуй хуудсанд ашигласан. Өөр газарт бол \ttitle командыг хэрэглэнэ
\thesistype{Бакалаврын төслийн ажил} % Удирдагчийн нэр, нүүр хуудсанд ашиглана. Дурын газарт бол \thesisname командыг хэрэглэнэ
\supervisor{Магистр Т.Золбоо} % Удирдагчийн нэр, нүүр хуудсанд ашиглана. Дурын газарт бол \supname командыг хэрэглэнэ
%\reader{Доктор (PhD) Ч.Цэнд-Аюуш} % Шүүмжлэгчийн нэр, Дурын газарт бол \readname командыг хэрэглэнэ
%\advisor{Магистр Б.Гүндсамбуу, Магистр Г.Цэнд-Аюуш} % Зөвлөгчийн нэр, Дурын газарт бол \advicename командыг хэрэглэнэ
\degreeind{D0613030000000014} % Мэргэжлийн индекс, нүүр болон хураангуй хуудсанд ашигласан. Өөр газарт бол \degreeid командыг хэрэглэнэ
\degree{Мэдээллийн систем} % Боловсролын зэрэг, нүүр болон хураангуй хуудсанд ашигласан. Өөр газарт бол \degreename командыг хэрэглэнэ
\authorshort{Д.Мөнхтүвшин} % Таны товч нэр, нүүр болон хураангуй хуудсанд ашигласан. Дурын газарт бол \shortname командыг хэрэглэнэ 
\authorlong{Давхарбаярын Мөнхтүвшин} % Таны бүтэн нэр, нүүр болон хураангуй хуудсанд ашигласан. Дурын газарт бол \longname командыг хэрэглэнэ 
\addresses{B150920009@mymust.net} % Таны хаяг, одоогоор ашиглаагүй. Өөр газарт бол \addressname командыг хэрэглэнэ
\subject{Мэдээллийн технологи} % Таны салбар, одоогоор ашиглаагүй. Дурынр газарт бол \subjectname командыг ашиглана
\keywords{шаблом} % Түлхүүр үгс, одоогоор ашиглаагүй. Дурын газарт бол \keywordnames командыг хэрэглэнэ
\university{\href{http://www.must.edu.mn}{Шинжлэх Ухаан Технологийн Их Сургууль}} % Их сургуулийн нэр ба веб хаяг. Дурын газарт бол \univname командыг хэрэглэнэ 
\department{Мэдээллийн технологийн салбар} % Сургууль/тэнхмийн нэр, нүүр болон хураангуй хуудсанд ашигласан. Дурын газарт бол\deptname командыг хэрэглэнэ
\deptchair{Проф. PhD. А.Эрдэнэбаатар} % Тэнхим/эрхлэгчийн нэр, нүүр болон хураангуй хуудсанд ашигласан. Дурын газарт бол\chairname командыг хэрэглэнэ
%\group{Робот техникийн баг} % Судалгааны баг/тэнхмийн нэр, нүүр хуудсанд ашигласан. Дурын газарт бол \groupname командыг хэрэглэнэ
\faculty{\href{http://www.sict.edu.mn}{Мэдээлэл, Холбооны Технологийн Сургууль}} % Салбар сургууль/факультетийн нэр, нүүр болон хураангуй хуудсанд ашигласан. Дурын газарт бол \facname командыг ашиглана

\hypersetup{pdftitle=\ttitle} % Pdf файлын гарчиг
\hypersetup{pdfauthor=\shortname} % Pdf файлын зохиогчийн нэр
\hypersetup{pdfkeywords=\keywordnames} % Pdf файлын түлхүүр үгс
\hypersetup{allcolors=black} % Pdf файлын бүх холбоос хар өнгөтэй

\renewcommand{\thesection}{\thechapter.\number\numexpr\value{section}-1\relax}
\renewcommand{\thesubsection}{\thesection.\number\numexpr\value{subsection}-1\relax}
\renewcommand{\thesubsubsection}{\thesubsection.\number\numexpr\value{subsubsection}-1\relax}
\setcounter{secnumdepth}{3}

\setcounter{chapter}{1}

\begin{document}

 % Агуулгын өмнөх хуудас дугаарлалт: i, ii, iii, iv... г.м.

\pagestyle{plain} % Тезисийн загварыг дуудах хүртэлх толгой мөрийн суурь загвар
\frontmatter
\include{FrontBackMatter/Titlepage} % Нүүр хуудас
%-------------------------------------------------------------------------------
%	ABSTRACT PAGE
%-------------------------------------------------------------------------------

\addchaptertocentry{Хураангуй} % Хураангуйг гарчигт нэмэх

\begin{center}
{\scshape\Large \univname\par} % Их сургуулийн нэр
{\scshape\large \facname\par}\vspace{0.5cm} % Их сургуулийн нэр
{\huge\textbf{{Хураангуй}} \par}
\bigskip
{\Large{\ttitle} \par} % Тезисийн нэр
\bigskip

{\normalsize \shortname \par} % Зохиогчийн нэр
\addressname
\end{center}

\textit{\textbf{Түлхүүр үгс: \keywordnames}}
\bigskip

Багш шалгалтыг оюутан эсвэл сурагчаар шаблом бөглүүлж авах бөгөөд тухайн шабломыг урьдаас манай application гаргаж өгсөн байх ба мөн application-оор шабломаа засна.



 % Ажлын хураангуй
%-------------------------------------------------------------------------------
%	ACKNOWLEDGEMENTS
%-------------------------------------------------------------------------------

\begin{acknowledgements}
\addchaptertocentry{\acknowledgementname}

\ldots

\end{acknowledgements}

 % Талархлын хуудас
%-------------------------------------------------------------------------------
%	ABBREVIATIONS
%-------------------------------------------------------------------------------

\begin{abbreviations}{ll} % Товчлолын жагсаалт оруулах (хоёр багатай хүснэгт)
\addchaptertocentry{\abbrevname}

\textbf{CPU} & \textbf{C}entral \textbf{P}processing \textbf{U}nit\\
\textbf{PHP} & \textbf{P}ersonal \textbf{H}ome \textbf{P}age\\
\textbf{АНУ} & \textbf{А}мерикийн \textbf{Н}эгдсэн \textbf{У}лс\\
\textbf{ЕБС} & \textbf{Е}рөнхий \textbf{Б}оловсролын \textbf{С}ургууль\\
\end{abbreviations}

 % Товчилсон үгс
\tableofcontents % Гарчиг хэвлэх
\newpage
\listoffigures % Зургийн жагсаалт хэвлэх
\newpage
\listoftables % Хүснэгтийн жагсаалт хэвлэх

\mainmatter


\section{Удиртгал}

    \subsection{Зорилго}
		Одоо байгаа үйл ажиллагааг электрон болгох ингэснээр хүссэн үедээ илүү хялбар хурданаар шалгалтын үр дүнг харах, нөөцөөс үл хамаарсан болгох, илүү дэлгэрэнгүй, найдвартай дахин ашиглагдах боломжтой болгоно.
    \subsection{Үйл ажиллагааны тодорхойлолт}
		Багш нар ихэнхдээ түлхүүр хүснэгт(шаблом) ашиглаж шалгалт авдаг. Энэ нь дараа засахад илүү хялбар байдаг. Багш тухайн хүүхдийн бөглөсөн шабломыгзөв бөглөсөн шабломтой тулгаж таар ч байгаа хариултуудыг тоолон дүн тавьдаг. Эсвэл эхнээс нь зөв буруу эсэхийг нь шалган зөвийн тоолж дүн тавьдаг.
    \subsection{Системийн ач холбогдол}
		Ямар асуултанд хамгийн олон хүүхэд зөв хариулсан байна, аль асууланд хамгийн их хариулж чадаагүй байна гэх зэрэг мэдээлэл багш шабломаа уншуулж дууссаны дараа гарч ирдэг. Энэ нь хүүхдүүд юуг түлхүү ойлгосон, юуг сайн ойлгоогүй гэх мэдээллийг багшид өгч чадна. Хүүхэд бүрийн асуулт түлхүүрүүд давхардахгүй дарааллаар гарч ирэх учир хуулах магадлал багасна.
	\subsection{Системийн хүрээ хязгаар}
	Системийн хэрэглэгч зөвхөн багш байх бөгөөд шалгалт үүсгэх, дүн засах, дүнгийн мэдээлэл харах, статистик мэдээлэл харах гэсэн үйлдлүүд байна.
% Бүлэг 1

\chapter{Багшийн журнал вэб системийн хөгжүүлэлт судалгааны хэсэг} 
% Бүлгийн нэр
\label{Chapter1} % Энэ бүлэг рүү ишлэл хийх бол \ref{Chapter1} командыг ашигла 

%-------------------------------------------------------------------------------

% Агуулгад ашигласан хэвшүүлэлтийн зарим командын тодорхойлолт
\newcommand{\keyword}[1]{\textbf{#1}}
\newcommand{\tabhead}[1]{\textbf{#1}}
\newcommand{\code}[1]{\texttt{#1}}
\newcommand{\file}[1]{\texttt{\bfseries#1}}
\newcommand{\option}[1]{\texttt{\itshape#1}}

%-------------------------------------------------------------------------------

\section{Хэрэглэгчийн судалгаа}
Энэхүү дипломын ажлын хүрээнд журнал дээр хийсэн судалгааг 2017 оны 09-р сарын 25-с 2017 оны 10-р сарын 23-ны өдөр хүртэл хийсэн судалгаа.
%-------------------------------------------------------------------------------
\begin{figure}[htbp]
	\centering
	\includegraphics[scale=0.05]{figures/20171006_161409.jpg}
	\centering
	\includegraphics[scale=0.05]{figures/20171006_162113.jpg}
	\centering
	\includegraphics[scale=0.05]{figures/20171006_162422.jpg}
	\caption[Хэрэглэгчийн судалгаа]{Сургууль дээр хэвлэгдэж ирдэг дэвтэр,журнал.}
	%\label{fig:Chart1}
\end{figure}

\begin{itemize}
\item Анги удирдсан багшийн дэвтэр(IV-XII анги)
\item Багшийн журнал (I-III анги)
\item Багшийн журнал (IV-XII анги)
\end{itemize}
	
\begin{enumerate}
	\item Анги удирдсан багшийн дэвтэр(IV-XII анги)
	\begin{enumerate}
		\item[1.1] Нүүр /1 нүүр/
		\item[1.2] Товьёг /1 нүүр/
		\item[1.3] Хуанли /1 нүүр/
		\item[1.4] Хичээлийн хуваарь /1 нүүр/
		\item[1.5] Журнал хөтлөх заавар /1 нүүр/
		\item[1.6] Суралцагчийн дэлгэрэнгүй бүртгэл /4 нүүр/
		\item[1.7] Хичээлээс гадуурх ажилд суралцагчийн оролцоо /4 нүүр/
		\item[1.8] Дугуйлан, секцэд хамрагдаж буй байдал /2 нүүр/
		\item[1.9] Суралцагчийн ололт амжилт /2 нүүр/
		\item[1.10] Эрүүл, зөв дадал хэвшил төлөвшүүлэх ажлын тэмдэглэл /4 нүүр/
		\item[1.11] Суралцагчийн төлөвшилд гарч буй өөрчлөлтийн талаарх тэмдэглэл /2 нүүр/
		\item[1.12] Багш, эцэг эхийн хамтын ажиллагаа /2 нүүр/
		\item[1.13] Сурах бичгийн бүртгэл /2 нүүр/
		\item[1.14] Ирцийн бүртгэл /8 нүүр/
		\item[1.15] Үнэлгээ 4 улирлаар /8 нүүр/
		\item[1.16] Жилийн эцсийн үнэлгээ /2 нүүр/
		\item[1.17] Шалгалтын дүн /1 нүүр/
		\item[1.18] Шилжилт хөдөлгөөн / Тасарсан хичээлийн цагийн тооцоо /1 нүүр/
		\item[1.19] Анги удирдсан багшийн тэмдэглэл /1 нүүр/\\ 
		\begin{flushright}
			{\large \textbf{ Нийт: 24 хуудас}}
		\end{flushright}
	\end{enumerate}
	\item Багшийн журнал (I-III анги)
	\begin{enumerate}
		\item[2.1] Нүүр /1 нүүр/
		\item[2.2] Товьёг /1 нүүр/
		\item[2.3] Хуанли /1 нүүр/
		\item[2.4] Хичээлийн хуваарь /1 нүүр/
		\item[2.5] Журнал хөтлөх заавар /1 нүүр/
		\item[2.6] Ирцийн бүртгэл /8 нүүр/
		\item[2.7] Үнэлгээ /86 нүүр/
		\item[2.8] Багшийн тэмдэглэл /86 нүүр/
		\item[2.9] Багшийн тэмдэглэл /1 нүүр/
		\item Анги удирдсан багшийн дэвтэр(I-III анги)
		\item[2.10] Нүүр /1 нүүр/
		\item[2.11] Товьёг /1 нүүр/
		\item[2.12] Журнал хөтлөх заавар /1 нүүр/
		\item[2.13] Суралцагчийн дэлгэрэнгүй бүртгэл /4 нүүр/
		\item[2.14] Хичээлээс гадуурх ажилд суралцагчийн оролцоо /4 нүүр/
		\item[2.15] Дугуйлан, секцэд хамрагдаж буй байдал /2 нүүр/
		\item[2.16] Суралцагчийн ололт амжилт /2 нүүр/
		\item[2.17] Эрүүл, зөв дадал хэвшил төлөвшүүлэх ажлын тэмдэглэл /4 нүүр/
		\item[2.18] Суралцагчийн төлөвшилд гарч буй өөрчлөлтийн талаарх тэмдэглэл /2 нүүр/
		\item[2.19] Багш, эцэг эхийн хамтын ажиллагаа /2 нүүр/
		\item[2.20] Сурах бичгийн бүртгэл /2 нүүр/
		\item[2.21] Шилжилт хөдөлгөөн / Тасарсан хичээлийн цагийн тооцоо /1 нүүр/
		\item[2.22] Анги удирдсан багшийн тэмдэглэл /1 нүүр/\\ 
		\begin{flushright}
			{\large \textbf{ Нийт: 65 хуудас}}
		\end{flushright}
	\end{enumerate}
	\item Багшийн журнал (IV-XII анги)
	\begin{enumerate}
		\item[2.1] Нүүр /1 нүүр/
		\item[2.2] Товьёг /1 нүүр/
		\item[2.3] Хуанли /1 нүүр/
		\item[2.4] Хичээлийн хуваарь /1 нүүр/
		\item[2.5] Журнал хөтлөх заавар /1 нүүр/
		\item[2.6] Ирц үнэлгээ /120 нүүр/
		\item[2.7] Багшийн тэмдэглэл /120 нүүр/
		\item[2.8] {Багшийн тэмдэглэл /1 нүүр/} \hfill \break
		\begin{flushright}
			{\large \textbf{ Нийт: 64 хуудас}}
		\end{flushright}
	\end{enumerate}
\end{enumerate}
%-------------------------------------------------------------------------------
%\section{}
Уг судалгааг Монгени цогцолбор сургууль, Ирээдүй цогцолбор 81-р сургууль, Сэтгэмж цогцолбор сургууль дээр хийгдсэн бөгөөд судалгааны явцад гарсан үр дүн дунджаар тавигдсан болно.
\begin{figure}[htbp]
	\centering
	\includegraphics[scale=0.8]{Chart/Chart1}
	\caption[Хэрэглэгчийн судалгаа]{2012-2017 оны хичээлийн жил тус бүрд нэг сургуульд дунджаар хэвлэгдэж ирэх журналын тоо(Судалгаа)}
	\label{fig:Chart2}
\end{figure}
\begin{figure}[htbp]
	\centering
	\includegraphics[scale=0.8]{Chart/Chart2}
	\caption[Хэрэглэгчийн судалгаа]{2012-2017 оны хичээлийн жил тус бүрд хэвлэгдсэн журнал, дэвтрийн хуудасны тоо (Судалгаа)}
	\label{fig:Chart2}
\end{figure}
\begin{figure}[htbp]
	\centering
	\includegraphics[scale=0.8]{Chart/Chart3}
	\caption[Хэрэглэгчийн судалгаа]{2012-2017 оны хичээлийн жил тус бүрд хэрэглэгдээгүй хуудасны тоо (Судалгаа)}
	\label{fig:Chart2}
\end{figure}\begin{figure}[htbp]
\centering
\includegraphics[scale=1]{Chart/Chart4}
\caption[Хэрэглэгчийн судалгаа]{2012-2017 оны хичээлийн жил тус бүрд нэг сургуульд Анги удирдсан багшийн дэвтэр(IV-XII)-ийн хэрэглэгдсэн, хэрэглэгдээгүй хуудасны тоо (Судалгаа)}
\label{fig:Chart2}
\end{figure}\begin{figure}[htbp]
\centering
\includegraphics[scale=1]{Chart/Chart5}
\caption[Хэрэглэгчийн судалгаа]{2012-2017 оны хичээлийн жил тус бүрд нэг сургуульд Багшийн журнал(IV-XII)-ийн хэрэглэгдсэн, хэрэглэгдээгүй хуудасны тоо (Судалгаа)}
\label{fig:Chart2}
\end{figure}\begin{figure}[htbp]
\centering
\includegraphics[scale=1]{Chart/Chart6}
\caption[Хэрэглэгчийн судалгаа]{2012-2017 оны хичээлийн жил тус бүрд нэг сургуульд Багшийн журнал (I-III)-ийн хэрэглэгдсэн, хэрэглэгдээгүй хуудасны тоо (Судалгаа)}
\label{fig:Chart2}
\end{figure}\begin{figure}[htbp]
\centering
\includegraphics[scale=0.9]{Chart/Chart7}
\caption[Хэрэглэгчийн судалгаа]{2012-2017 оны хичээлийн жил тус бүрд Нийслэлийн хэмжээнд Анги удирдсан багшийн дэвтэр(IV-XII)-ийн хэрэглэгдсэн, хэрэглэгдээгүй хуудасны тоо (Судалгаа)}
\label{fig:Chart2}
\end{figure}\begin{figure}[htbp]
\centering
\includegraphics[scale=0.9]{Chart/Chart8}
\caption[Хэрэглэгчийн судалгаа]{2012-2017 оны хичээлийн жил тус бүрд Нийслэлийн хэмжээнд Багшийн журнал (I-III)-ийн хэрэглэгдсэн, хэрэглэгдээгүй хуудасны тоо (Судалгаа)}
\label{fig:Chart2}
\end{figure}\begin{figure}[htbp]
\centering
\includegraphics[scale=0.9]{Chart/Chart9}
\caption[Хэрэглэгчийн судалгаа]{2012-2017 оны хичээлийн жил тус бүрд Нийслэлийн хэмжээнд Багшийн журнал(IV-XII)-ийн хэрэглэгдсэн, хэрэглэгдээгүй хуудасны тоо (Судалгаа)}
\label{fig:Chart2}
\end{figure}\begin{figure}[htbp]
\centering
\includegraphics[scale=0.7]{Chart/Chart10}
\caption[Хэрэглэгчийн судалгаа]{2012-2017 оны хичээлийн жил тус бүрд Нийслэлийн хэмжээнд хэрэглэгдээгүй хуудасны тоо(Судалгаа)}
\label{fig:Chart2}
\end{figure}
\section{Судалгааны дүгнэлт}
Энэхүү судалгааны үр дүнд хэрэглэгдээгүй цаасны хэмжээ 650000 орчим гарсан бөгөөд багшийн журнал 5 жил, Анги удирдсан багшийн дэвтэр 20 жилийн хугацаанд хэрэглэгдэн цахим болж устгагддаг байна. Одоогийн байдлаар програм хангамжийн байгууллагууд төрөл бүрийн цахим журнал хийж байгаа боловч бүрэн орлож чадахгүй байна. Иймд зөвхөн журнал гэдэг зүйлийг дангаар авч үзэж цахимжуулах арга хэмжээ авах хэрэгтэй.\\
500 ширхэг бичгийн цаас хийхэд дунджаар 2,6 кг мод хэрэглэгддэг бол Нийслэлийн хэмжээнд нийт сургуулиудын хэмжээнд 3,3 тонн цаас хий дэмий үрэгддэг гэсэн үг юм. 
%-------------------------------------------------------------------------------

\section{Технологийн судалгаа}

\section{MySQL}
MySQL нь холбоост өгөгдлийн санг удирдах систем юм. MySQL хэмээх нэрний хувьд уг системийг санаачлан хөгжүүлэгч Micheal Widenius-ын охины нэр My + SQL(Structed Query Language) гэсэн утгатай ажээ.
Энэ систем нь GNU (General Public License) буюу нээлтэй эхийн систем учир хүссэн хэн бүхэн хөгжүүлэлтэнд оролцож, үнэгүй хэрэглэж болох юм. Эзэмшигч нь алдарт Java-г хөгжүүлсэн Sun MicroSystems компани байсан ба, одоогоор Sun-г Oracle корпораци эзэмших болсон билээ.
Үнэгүй програм хангамжийн өгөгдлийн санг удирдах системд ихэвчлэн MySQL-ийг хэрэглэдэг бөгөөд тэдгээрийн сонгодог жишээ гэвэл Joomla, Drupal, Wordpress, phpBB гэх мэт агуулга удирдах системүүд (CMS-Content Management System), Wikipedia, Facebook, Google гэх мэт томоохон компаниуд хэрэглэдэг юм.
Хөгжүүлэлт нь C/C++ хэл дээр хийгдсэн ба AIX, BSDi, FreeBSD, HP-UX, i5/OS, Linux, Mac OS X, NetBSD, Novell NetWare, OpenBSD, OpenSolaris, eComStation, OS/2 Warp, QNX, IRIX, Solaris, Symbian, SunOS, SCO OpenServer, SCO UnixWare, Sanos, Tru64, Microsoft Windows гэсэн олон үйлдлийн системүүд дээр ажилладаг.
MySQL бол хамгийн өргөн хэрэглээтэй нээлттэй эхийн (Open Source) өгөгдлийн сан удирдах програм юм. Анх 1995 онд зах зээлд гарсан ба с/с++ хэл дээр бичигдсэн. Одоогийн байдлаар 5.7 нь хамгийн сүүлийн хувилбар болон гараад байна. Энэ сүүлийн хувилбар дээр нэмэгдсэн давуу талууд гэвэл 3 дахин хурдан үйл ажиллагаатай болсон мөн натив JSON дэмжигчтэй болсон гэх мэт шинэлэг үйлдлүүд нэмэгдсэн байна.

\section{Php}
  Rasmus Lerdorf WWW-д вэб хуудас үүсгэх үедээ өгөгдөл боловсруулах хялбархан арга хайж байгаад 1995 онд PHP хэлийг скрипт хэл байдлаар зохиосон.
PHP нь сервер талын скрипт хэл ба динамик вэб хуудас хийхэд илүү тохиромжтой. Энэ скрипт хэл нь энгийн хэрэглээний вэб сайтаас эхлээд байгууллагын иж бүрэн вэб программ хийж болохоор MySQL мэтийн өгөгдлийн сантай харилцан ажиллах боломжтой.
Хуудас ачаалах үед броузерээр нэг бүрчлэн уншдаг HTML-тэй адилгүй, PHP баримтыг бэлтгэхдээ серверээр урьдчилан боловсруулдаг. PHP код агуулсан хуудас нь хэрэглэгчийн броузерт илгээгдхээс өмнө серверээр боловсруулагдсан байдаг.
PHP хэлний өөр нэг давуу тал бол скриптэн хэл юм. Ихэнх програмчлалын хэлнүүдэд ажиллахын өмнө машины хэл рүү хөрвүүлэх тусгай файлууд /compile/ шаардлагатай байдаг бол PHP хэлний хувьд хөрвүүлэлт хийх шаардлагагүй байдаг тул код засварлах болон шалгахад илүү хурдан байдаг

\section{JQuery}
2006 оны эхээр АНУ-ын Нью-Иорк хотын BarCamp-д John Resig хэмээх вэб хөгжүүлэгч залуу jQuery сангийн тухай анх мэдэгджээ. Resig өөрийн вэб сайтдаа: Тухайн үед байгаа сангуудад сэтгэл дундуур байгаагаа, мөн түүнчлэн JavaScript – ий тухай бичилтийг нь багасгаснаар маш их ажил хөнгөвчлөх боломжтой, энгийн үйлдлүүдэд зориулан тусгай хэрэгслүүд нэмэх хэрэгтэй гэж дурдсан байдаг.
Хөгжүүлэх нийгэмлэгт jQuery нь томоохон амжилт авчирсан төдийгүй улмаар маш хурдтай хөгжсөн. Бусад хөгжүүлэгчид сан боловсронгуй болгоход тусалж эхэлснээр jQuery – гийн анхны хувилбар 1.0 нь 2006 оны 8-р сарын 26- нд гарсан.
Түүнээс хойш jQuery 3.1.1 хувилбар гарсан ба хөгжүүлэлтийн нийгэмлэгээс plug-in –ийг маш ихээр оруулсан. Plug-in нь jQuery – ийн сангийн цөм хэсэг биш харин нэмэлт хэрэгсэл юм. 
jQuery – гийн давуу талууд нь:
\begin{itemize}
\item Файлын хэмжээ бага
\item Маш энгийн бичилттэй
\item Холбоо бүхий method – уудтай
\item Санг өргөтгөх plug-in нь энгийн бүтэцтэй
\item Асар том онлайн нийгэмлэгтэй
\item JQueryUI мэтийн jQuery – гийн нэмэлт сонголтуудтай
\end{itemize}
\section{Service}
2006 оны эхээр АНУ-ын Нью-Иорк хотын BarCamp-д John Resig хэмээх вэб хөгжүүлэгч залуу jQuery сангийн тухай анх мэдэгджээ. Resig өөрийн вэб сайтдаа: Тухайн үед байгаа сангуудад сэтгэл дундуур байгаагаа, мөн түүнчлэн JavaScript – ий тухай бичилтийг нь багасгаснаар маш их ажил хөнгөвчлөх боломжтой, энгийн үйлдлүүдэд зориулан тусгай хэрэгслүүд нэмэх хэрэгтэй гэж дурдсан байдаг.
Хөгжүүлэх нийгэмлэгт jQuery нь томоохон амжилт авчирсан төдийгүй улмаар маш хурдтай хөгжсөн. Бусад хөгжүүлэгчид сан боловсронгуй болгоход тусалж эхэлснээр jQuery – гийн анхны хувилбар 1.0 нь 2006 оны 8-р сарын 26- нд гарсан.
Түүнээс хойш jQuery 3.1.1 хувилбар гарсан ба хөгжүүлэлтийн нийгэмлэгээс plug-in –ийг маш ихээр оруулсан. Plug-in нь jQuery – ийн сангийн цөм хэсэг биш харин нэмэлт хэрэгсэл юм. 
jQuery – гийн давуу талууд нь:
\begin{itemize}
\item Файлын хэмжээ бага
\item Маш энгийн бичилттэй
\item Холбоо бүхий method – уудтай
\item Санг өргөтгөх plug-in нь энгийн бүтэцтэй
\item Асар том онлайн нийгэмлэгтэй
\item JQueryUI мэтийн jQuery – гийн нэмэлт сонголтуудтай
\end{itemize}
\section{Бүлгийн дүгнэлт}
%-------------------------------------------------------------------------------

\section{Шинжилгээ }
\subsection{Үйл ажиллагааны диаграм}
\includegraphics[width=15cm,height=6cm, scale=0.5]{Figures/ac1.png}
\hspace*{0pt}\hfill Багш шаблом урьчилан бэлдэж шалгалт авах үйл ажиллагааны диаграм.
\newline
\newline

\includegraphics[width=15cm,height=6cm, scale=0.5]{Figures/ac2.png}
\hspace*{0pt}\hfill Багш шаблом урьчилан бэлдээгүй шалгалт авах үйл ажиллагааны диаграм.
\newline
\subsection{Функциональ шаардлага}
\begin{flushleft}
- Нэвтрэх
\linebreak
- Асуулт бүртгэх
\linebreak
- Хариулт бүртгэх
\linebreak
- Шалгалт бүртгэх\linebreak
- Асуултын жагсаалт харах\linebreak
- Шалгалтын үр дүнгийн тайлан харах.\linebreak
- Шалгалт өгсөн оюутнуудын тоогоор тайлан харах.\linebreak
\end{flushleft}
\subsection{Функциональ бус шаардлага}
\begin{flushleft}
- Шалгалтын асуултууд бүгд давхардахгүй дараалалтай шалгалтууд хэвлэгдэнэ.\linebreak
- Шалгалтын ID болон Оюутны ID-г шабломноос авна.\linebreak
- Оюутны шалгалтын оноог бааз руу хуулна.\linebreak
- Шалгалтын асуултууд дээр дүн шинжилгээ хийнэ.\linebreak
\end{flushleft}

% Бүлэг 3

\section{\ttitle-н Зохиомж} % Зарим нэг зөвлөмж


	\subsection{Системийн үйл ажиллагааны тухай}

%	SECTION
%-------------------------------------------------------------------------------
	\subsection{Системийг ашиглах хэрэглэгчид}
		\subsubsection{Системийн оролцогч}
    	\subsubsection{Системийн тоглогч}
  
%	SECTION
%-------------------------------------------------------------------------------
	\subsection{Функцийн шаардлага}

%	SECTION
%-------------------------------------------------------------------------------
	\subsection{Функцийн бус шаардлага}

%	SECTION
%-------------------------------------------------------------------------------  
	\subsection{Юзкейс диаграм}

%	SECTION
%-------------------------------------------------------------------------------  
	\subsection{Юзкейс тодорхойлолт}

%	SECTION
%-------------------------------------------------------------------------------
	\subsection{Класс диаграм}

%	SECTION
%-------------------------------------------------------------------------------
	\subsection{Дарааллын диаграм}

%	SECTION
%-------------------------------------------------------------------------------
	\subsection{Үйл ажиллагааны диаграм}

%	SECTION
%-------------------------------------------------------------------------------
	\subsection{Бүлгийн дүгнэлт}

\pagestyle{plain}
% Ерөнхий дүгнэлт

\phantomsection
\addchaptertocentry{Ерөнхий дүгнэлт}
\label{Summary} % Энэ бүлэг рүү ишлэл хийх бол
%-------------------------------------------------------------------------------
%	Summary
%-------------------------------------------------------------------------------
\section*{Ерөнхий дүгнэлт}



\end{document}