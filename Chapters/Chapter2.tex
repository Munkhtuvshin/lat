% Chapter Template

\section{Хөгжүүлэлтэд ашиглах технологи, хэрэгслүүд} % Main chapter title

\label{Chapter2} % Change X to a consecutive number; for referencing this chapter elsewhere, use \ref{ChapterX}

%----------------------------------------------------------------------------------------
%   SECTION 1
%----------------------------------------------------------------------------------------
\section{Технологийн тухай}
    \subsection{Технологи, прэймворк болон АПИ -ийн тухай үндсэн ойлголт}
    \subsection{Тэдгээрийн давуу ба дутагдалтай талууд}
    \subsection{Бусад ижил төстэй технологи, прэймворк, АПИ ба тэдгээртэй харьцуулсан судалгаа}
    \subsection{Технологийн сонголт дээрх SWOT шинжилгээ}
\section{Технологийн ерөнхий шаардлага}
    \subsection{Үйлдлийн системийн шаардлага}
    \subsection{Програмчлалын хэлний шаардлага}
    \subsection{Суулгах програмчлалын хэрэгслүүд}
    \subsection{Өгөгдлийн сангийн шаардлага}
\section{Суурилуулах заавар}
    \subsection{Сервер талд суурилуулах заавар}
    \subsection{Клиент талд суурилуулах заавар}
    \subsection{Програмчлалын хэрэгсэлтэй холбох заавар}
\section{Функц үйлдлүүд, тэдгээрийг хэрэглэх тухай}
%Класс, функц, массив, өгөгдлийн сантай холбох, өгөгдлийн сангаас өгөгдөл авах, бичих, устгах ... гэх мэтийг нэг бүрчлэн хэрхэн хэрэглэж байгааг тайлбарлах, жишээ гаргах
    \subsection{...}
    \subsection{...}