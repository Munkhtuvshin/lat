
\section{Шинжилгээ }
\subsection{Үйл ажиллагааны диаграм}
\includegraphics[width=15cm,height=6cm, scale=0.5]{Figures/ac1.png}
\hspace*{0pt}\hfill Багш шаблом урьчилан бэлдэж шалгалт авах үйл ажиллагааны диаграм.
\newline
\newline

\includegraphics[width=15cm,height=6cm, scale=0.5]{Figures/ac2.png}
\hspace*{0pt}\hfill Багш шаблом урьчилан бэлдээгүй шалгалт авах үйл ажиллагааны диаграм.
\newline
\subsection{Функциональ шаардлага}
\begin{flushleft}
- Асуултын санд асуулт оруулна (1 сонголттой, олон сонголттой гэсэн төрөлтэй байна.)
\linebreak
- Шалгалтын материал боловсруулна (Хүүхдийн тоо асуултын тоо гэх мэтээр)
\linebreak
- Шалгалтын материалууд хэвлэнэ.\linebreak
- Шалгалт засна.\linebreak
- Шалгалтын үр дүнгийн статистик харна.
- Шалгалтын үр дүн харна.\linebreak
\end{flushleft}
\subsection{Функциональ бус шаардлага}
\begin{flushleft}
- Шалгалтын асуултууд бүгд давхардахгүй дараалалтай шалгалтууд хэвлэгдэнэ.\linebreak
- Шалгалтын ID болон Оюутны ID-г шабломноос авна.\linebreak
- Оюутны шалгалтын оноог бааз руу хуулна.\linebreak
- Шалгалтын асуултууд дээр дүн шинжилгээ хийнэ.\linebreak
\end{flushleft}
