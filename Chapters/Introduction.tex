
\section{Удиртгал}

    \subsection{Зорилго}
		Одоо байгаа үйл ажиллагааг электрон болгох ингэснээр хүссэн үедээ илүү хялбар хурданаар шалгалтын үр дүнг харах, нөөцөөс үл хамаарсан болгох, илүү дэлгэрэнгүй, найдвартай дахин ашиглагдах боломжтой болгоно.
    \subsection{Үйл ажиллагааны тодорхойлолт}
		Багш нар ихэнхдээ түлхүүр хүснэгт(шаблом) ашиглаж шалгалт авдаг. Энэ нь дараа засахад илүү хялбар байдаг. Багш тухайн хүүхдийн бөглөсөн шабломыгзөв бөглөсөн шабломтой тулгаж таар ч байгаа хариултуудыг тоолон дүн тавьдаг. Эсвэл эхнээс нь зөв буруу эсэхийг нь шалган зөвийн тоолж дүн тавьдаг.
    \subsection{Системийн ач холбогдол}
		Ямар асуултанд хамгийн олон хүүхэд зөв хариулсан байна, аль асууланд хамгийн их хариулж чадаагүй байна гэх зэрэг мэдээлэл багш шабломаа уншуулж дууссаны дараа гарч ирдэг. Энэ нь хүүхдүүд юуг түлхүү ойлгосон, юуг сайн ойлгоогүй гэх мэдээллийг багшид өгч чадна. Хүүхэд бүрийн асуулт түлхүүрүүд давхардахгүй дарааллаар гарч ирэх учир хуулах магадлал багасна.