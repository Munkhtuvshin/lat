% Бүлэг 1


%-------------------------------------------------------------------------------

% Агуулгад ашигласан хэвшүүлэлтийн зарим командын тодорхойлолт
\newcommand{\keyword}[1]{\textbf{#1}}
\newcommand{\tabhead}[1]{\textbf{#1}}
\newcommand{\code}[1]{\texttt{#1}}
\newcommand{\file}[1]{\texttt{\bfseries#1}}
\newcommand{\option}[1]{\texttt{\itshape#1}}

\section{Судалгаа}	
\subsection{Системийн судалгаа}
	Энэ төрлийн ситемүүд нь computer-ийн ухааны computer vision-ы салбар дээр үндэслэн хийгддэг бөгөөд хиймэл оюун, машин сургалттай хослон ажилдаг тэгсэнээр илүү нарийн үр дүнд хүч чаддаг. Computer vision чиглэлийн OpenCV Фреймворк нь 1999 оноос албан ёсоор нээлтээ хийж хөгжүүлэгдэж эхлэсэн хамгийн том фреймворк юм. eyeGrade програм нь хамгийн олон хэрэглэгчдэд хүрсэн desktop програм юм. Mobile төхөөрөмжийн хувьд zipgrade гэх application-ыг гадаадад ашигладаг мөн kahook гээд арай өөр төрлийн илүү өргөн хүрээг хамарсан application маш их ашиглагддаг.gradepen
\subsection{Технологийн судалгаа}
	\subsubsection{React Native}
	RN нь javascript дээр бичигдсэн ios болог android application хөгжүүлэхд зориулагдсан hybrid фреймворк юм. RN нь react javascript дээр тулгуурладаг. React - ийн component level pattern нь системийг өргөтгөхөд илүү хялбар уян хатан болгодог.
	\subsubsection{OpenCV}
	Opencv нь с болон с++  дээр бичигдсэн мөн бүх төрлийн cpu-дээр ажилладаг, бараг бүх түгээмэл програмчлалын хэл дээр байдаг. Cannyedge болон otsi-ийн илрүүлэлтийн алгоритмуудыг зэрэг маш олон хэрэгцээ ихтэй сангуудыг агуулдаг
	\subsubsection{MongoDB}
	MongoDB өгөгдөлийн сан нь харьцаа өгөдөлийн сан биш бөгөөд өгөдөлийн давхардалт зэрэг зүйлийг шалгадаггүй. Давуу тал нь илүү хурдан тооцоолох чадвартай. Бүх өгөгдөл collection-д хадгалагддаг.
	\subsubsection{Python}
	Python нь бүтээмж өндөртэй програмчлалын хэл юм.
	\subsection{Бүлгийн дүгнэлт}
	Python нь OpenCV-тэй сайн зохицож ажилладаг бөгөөд бичиглэл ойлгомжтой гэдэг үйднээс сонгосон. MongoDB бол харьцаа өгөгдөлийн сангаас хамаагүй уян хатан өөрчлөгдөх боломж хялбар бөгөөд өргөтгөхөд хялбар энгийн учир сонгосон. React-native ч мөн адил өргөтгөх өөрчлөх илүү хялбар гэдэг үүднээс сонгосон.
